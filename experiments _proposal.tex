\section{Proposed Experiments: Quantum Embeddings for Theory of Mind}
\label{sec:experiments-prose}

We propose a suite of experiments designed to test whether quantum-inspired and quantum-native embeddings can advantage Theory of Mind (ToM) observers operating in partially observable grid-worlds. Throughout this section, we refer to the observer’s belief representation as the \emph{current context}, emphasizing its role as a structured, temporally aggregated summary of the agent–environment interaction history rather than a static hidden state. Our aim is to evaluate whether parameterized quantum circuits (PQCs), used either alone or in hybrid conjunction with classical neural components, enable richer transformations over this current context, thereby improving performance on ToM judgments—most notably those that require counterfactual, perspective-sensitive reasoning such as false-belief tasks.

The experimental program builds directly on our existing analyses that isolate two loci of representational choice: the \emph{state encoder}, which maps raw observations and features into a compact latent representation, and the \emph{current context module}, which serves as the mental net’s input by integrating information across time for ToM inference. In one strand of experiments, we keep the encoder fixed and vary the nature of the current context module, contrasting a fully classical implementation against quantum and hybrid alternatives. In another strand, we invert the configuration: we hold the current context machinery classical while endowing the state encoder with quantum or hybrid capabilities in order to isolate where, along the perception–inference pipeline, quantum feature maps yield the greatest marginal benefit. Finally, we examine a dual setting in which both the encoder and the current context are equipped with quantum or hybrid embeddings, allowing us to probe compositional effects and potential synergies that might not be visible under single-site substitutions.

All models are trained and evaluated within the same grid-world environment under matched conditions—identical grid sizes, fields of view, episode lengths, and agent policy mixtures—so that any observed differences can credibly be attributed to representational changes rather than data or optimization confounds. We adopt a consistent training protocol across conditions, including the same optimizer, learning rate, batch size, and early-stopping criterion based on validation accuracy with a fixed patience window. To guard against idiosyncrasies of initialization and data ordering, we repeat each condition across multiple random seeds and report aggregate statistics. Where quantum toolchains are unavailable, we fall back to classical baselines and a hybrid configuration; for reproducibility, we also support plotting from fixed JSON result files that capture the full set of summary metrics.

Evaluation focuses on several complementary perspectives. We report overall validation accuracy on ToM queries alongside decomposition into false-belief and visible sub-tasks, the latter serving as a sanity check that gains are not restricted to trivial cases solvable from immediately observable information. Beyond accuracy, we quantify the computational footprint of each approach by measuring total training time and average per-epoch time, and we track parameter counts to assess whether quantum and hybrid configurations achieve improvements through more efficient use of capacity. To better understand data efficiency, we measure accuracy as a function of the number of training episodes, and to probe the structure of the quantum contribution itself, we conduct scaling studies that vary the number of qubits and circuit depth (including entanglement topology), recording how accuracy and runtime co-vary with these design choices.

A particular emphasis of this investigation is the role of the \emph{mental net} that operates over the current context. We hypothesize that placing quantum or hybrid embedding capacity directly at this locus is especially beneficial for false-belief reasoning, as the mental net must reconcile conflicting temporal evidence and represent counterfactual states—conditions under which interference-enabled feature maps may offer a distinct advantage. We therefore include targeted comparisons that keep the state encoder classical while swapping the mental net’s embedding between classical, quantum, and hybrid variants, holding all else constant. In parallel, mirror-image comparisons at the encoder level allow us to test whether quantum benefits are predominantly perceptual (at the encoding stage) or inferential (within the current context dynamics), or whether the two interact.

Taken together, these experiments will provide a cohesive picture of where and how quantum embeddings matter within ToM observers. By triangulating accuracy on challenging false-belief cases, computational and parameter efficiency, sample-efficiency curves, and qubit/depth scaling behavior, we aim to distinguish genuine representational advantages from mere increases in computational budget. If the hypotheses are borne out, the results will demonstrate that quantum or hybrid transformations over the current context—and especially within the mental net—can enhance perspective-sensitive inference in a way that is both measurable and practically realizable under realistic training constraints.

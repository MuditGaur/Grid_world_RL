\section{Proposed Experiments: Quantum Embeddings for Theory of Mind}
\label{sec:experiments}

\subsection{Models and Embedding Variants}
We investigate whether quantum-inspired and quantum-native embeddings can improve Theory of Mind (ToM) inference by operating over what we term the \emph{current context}, a temporally aggregated latent representation that replaces the traditional notion of a belief state. Our study considers two loci where representational choices can matter most: the state encoder that transforms raw grid-world observations into a compact latent, and the module that consumes the current context---the \emph{mental net}---to render ToM judgments. At each locus we contrast classical neural implementations with quantum and hybrid alternatives. Quantum variants employ parameterized quantum circuits with amplitude or angle encodings and trainable rotations interleaved with entangling gates, while hybrid designs situate a quantum feature map within classical pre- and post-processing layers. By systematically enabling quantum or hybrid capacity at the encoder, at the mental net, or at both, we seek to disentangle perceptual from inferential gains and to test for compositional benefits when quantum transformations are applied across the pipeline.

\subsection{Tasks and Environment}
All models are evaluated in a partially observable grid-world where agents pursue goals under occlusion and where an external observer must infer latent mental states and predictions. The observer has access to consistent input features derived from environment rollouts and must answer ToM queries that we partition into overall accuracy, accuracy on false-belief cases that require counterfactual perspective-taking, and accuracy on visible cases that can be resolved from immediately available information. To ensure that comparisons reflect representational differences rather than dataset artifacts, we generate matched rollouts across conditions with identical grid sizes, fields of view, episode lengths, agent policy mixtures, and random seeds.

\subsection{Experimental Factors and Conditions}
Our comparisons follow three complementary strands. First, we vary the current context module while holding the state encoder fixed, contrasting classical, quantum, and hybrid mental nets to ask whether quantum feature maps over temporally aggregated representations selectively help with perspective-sensitive reasoning. Second, we invert the configuration by fixing the current context module to a classical implementation and equipping the state encoder with quantum or hybrid embeddings; this isolates potential advantages at the perceptual front-end. Third, we explore a dual quantum setting in which both the encoder and the current context module adopt quantum or hybrid embeddings, testing for synergistic effects that may only emerge when quantum transformations are composed. To better understand capacity and cost trade-offs, we conduct scaling studies that vary the number of qubits, circuit depth, and entanglement topology, and we include an input-modality control in which the model consumes either the agent’s partial observation or observer-centric features.

\subsection{Training Protocol and Early Stopping}
All conditions are trained under a matched optimization protocol. We use the same optimizer, learning rate, batch size, and patience-based early stopping on validation accuracy so that training dynamics are comparable across representational choices. Each configuration is repeated over multiple random seeds to average over initialization and data ordering effects, and per-epoch logs of losses and sub-task accuracies are retained to identify the best validation epoch. This protocol mirrors our existing analyses and ensures that any differences we report can be attributed to the embeddings rather than to training idiosyncrasies.

\subsection{Evaluation Metrics and Reporting}
We report validation accuracy at the best epoch both in aggregate and decomposed into false-belief and visible subsets. To quantify computational overheads, we record total training time and the average time per epoch, and we tally parameter counts to assess parameter efficiency. We further characterize sample efficiency by tracing accuracy as a function of the number of training episodes. For quantum and hybrid variants, we couple these metrics with scaling curves that track how accuracy and runtime evolve with qubit count and circuit depth. Results are communicated with concise six-panel visual summaries---covering accuracies, runtimes, and parameter counts---and with tables that present per-condition means and variability across seeds.

\subsection{Hypotheses}
We hypothesize that quantum and hybrid embeddings will confer the greatest benefits when placed at the locus of inference over the current context, that is, within the mental net. Because false-belief judgments require reconciling conflicting temporal evidence and maintaining counterfactual representations, interference-enabled quantum feature maps may unlock representational structures that classical networks approximate only with greater depth or parameter count. We also expect hybrid designs to approach the accuracy of fully quantum variants while reducing runtime and parameter costs, offering a pragmatic path where quantum resources are limited. When quantum embeddings are applied both at the encoder and at the mental net, we anticipate compositional gains beyond either substitution alone, up to a point determined by circuit depth and hardware-effective qubit counts. Finally, we expect accuracy to improve as qubits and entanglement increase, with diminishing returns that clarify compute–performance trade-offs.

\subsection{Reproducibility and Ablations}
To facilitate reproducibility, we hold environment, data-generation, and training hyperparameters constant across conditions and provide fixed-result JSON files from which plots and tables can be regenerated. We ablate the placement of quantum capacity (encoder-only versus mental-net-only), explore alternative entanglement patterns and depths, vary the input modality to test robustness under occlusion, and adjust early-stopping patience to ensure results are not artifacts of training schedules. Where quantum toolchains are unavailable, we provide classical baselines and hybrid fallbacks to maintain continuity across experimental sites.

\subsection{Expected Outcomes and Impact}
If our hypotheses are correct, the results will show that quantum or hybrid transformations over the current context can enhance ToM performance, particularly on tasks that demand perspective-sensitive reasoning such as false-belief cases, while quantifying the runtime and parameter costs required to realize these gains. Such findings would support the use of quantum embeddings in practical ToM observers by identifying configurations---notably quantum or hybrid mental nets---that deliver measurable benefits under realistic computational budgets. By coupling accuracy, efficiency, and scaling analyses, the proposed experiments aim to separate genuine representational advantages from mere increases in capacity, thereby grounding claims about the role of quantum embeddings in cognitively motivated settings.
